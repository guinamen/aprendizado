\section{Conclusão e Trabalhos Futuros}


\begin{frame}

    \frametitle{Respostas}

    \begin{questions}
        \item Os algoritmos de mineração de dados mais úteis para descrever uma base de dados foram: FPMax, OpusMiner, FHM Freq, FHN e Descoberta de Subgrupo. Esta sequência de algoritmos descrevem os dados de forma mais genérica à mais específica.
        \item Eles conseguem, em sequência, gerarem um conjunto de informações mais genéricas até as mais específicas.
        \item A melhor sequência de execução dos algoritmos foi a apresentada no trabalho, a grande descoberta do trabalho foi que para utilizar o algoritmo de descoberta de subgrupos de forma mais eficiênte, a melhor forma, é compreender a base de dados para segmentar muito bem os registros que serão processados.
\end{questions}

\end{frame}


\begin{frame}

    \frametitle{Conclusão}

    O trabalho demonstrou a eficacia de uma arquitetura de dutos e filtros com algoritmos de mineração de dados para descrever o espaço urbano de Belo Horizonte de forma compacta, revelando padroes e tendências no uso e ocupação do solo a partir de dados imobiliarios.

\end{frame}


\begin{frame}

    \frametitle{Trabalhos Futuros}

	Destaca-se a incorporação de dados espaço-temporais para análise de séries históricas.
	\newline
	\newline
	A integração com outras fontes de dados, como redes sociais e imagens de satélite.
	\newline
	\newline
	E uso de outras técnicas de mineração de dados como a mineração de dados de mesma localidade, para evitar a segmentação por regionais.
\end{frame}


\begin{frame}
    \begin{center}
        \Huge{OBRIGADO!}
    \end{center}
\end{frame}
